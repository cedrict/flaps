 \subsection{allocate\_mesh\_arrays}
 this subroutine allocates the mesh variable to the nel size 
 and estimates how much memory the mesh is going to take.
 \subsection{NNN}
 Basis functions ${\bN}_i$. Spaces supported: $Q_1$, $Q_2$.
 \subsection{dNNNdr}
 Basis functions derivatives $\partial{\bN}_i/\partial r$. Spaces supported: $Q_1$, $Q_2$.
 \subsection{dNNNds}
 Basis functions derivatives $\partial{\bN}_i/\partial s$. Spaces supported: $Q_1$, $Q_2$.
 \subsection{compute\_dNdx\_dNdz\_at\_qpt.f90}
 This subroutine computes $\partial{\bN}/\partial \{x,z\}$, 
 at a quadrature point iq (between 1 and nqel)) passed as argument.
 It also returns the determinant jcob of the Jacobian matrix.
 \subsection{compute\_normals}

 \subsection{geometry\_setup}

 \subsection{output\_qpoints}
 This subroutine exports all quadrature points (and the fields they carry)
 in a vtu file in the OUTPUT folder.
 \subsection{output\_solution\_to\_ascii}
 This subroutine export the nodal fields to the file {solution.ascii} in the OUTPUT folder.
 \subsection{output\_solution\_to\_vtu}
 This subroutine exports the solution field and derivatives in vtu format in a file found in 
 the OUTPUT folder. 
 \subsection{quadrature\_setup.f90}
 This subroutine computes/fills all quadrature-related arrays for each element.
 It further computes the real coordinates $(x_q,y_q,z_q)$ and reduced 
 coordinates $(r_q,s_q,t_q)$ of these points, and assigns them their weights.
 The required constants for the quadrature schemes are in 
 {\filenamefont module\_quadrature.f90}.
 Note that JxWq only receives Wq and will be multiplied by J in the sanity subroutine.
 \subsection{setup\_A}
 Matrix A is the assembled FE matrix.
 If the geometry is {\tt 'box'} or {\tt 'chunk'} the number of nonzeros in the matrix and 
 its sparsity  structure are computed in a very efficient way. 
 \subsection{template}

 \subsection{test\_basis\_functions}
 This subroutine tests the consistency of the basis functions. 
 An analytical velocity field is prescribed (constant, linear or quadratic) and the 
 corresponding values are computed onto the quadrature points via the 
 (derivatives of the) basis functions.
 It generates three ascii files in the {\foldernamefont OUTPUT} folder.
