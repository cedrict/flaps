\documentclass[a4paper,12pt]{article}
\usepackage[cm]{fullpage}
\usepackage{url}
\usepackage{xcolor}
\usepackage{graphicx}
\usepackage{amsmath}
\usepackage{amsfonts}
\usepackage{upgreek}
\usepackage{bm}
\usepackage{listings}
\usepackage{tikz}

\lstset{ 
  language=Python,
  backgroundcolor=\color{white},   % choose the background color; you must add \usepackage{xcolor}; should come as last argument
  basicstyle=\footnotesize,        % the size of the fonts that are used for the code
  breakatwhitespace=false,         % sets if automatic breaks should only happen at whitespace
  breaklines=true,                 % sets automatic line breaking
  captionpos=b,                    % sets the caption-position to bottom
  frame=single,                    % adds a frame around the code
  keepspaces=true,                 % keeps spaces in text, useful for keeping indentation of code 
  keywordstyle=\color{blue},       % keyword style
}

\usetikzlibrary{arrows, arrows.meta}

\bibliographystyle{plain} % We choose the "plain" reference style

\newcommand{\nn}{\nonumber}
\newcommand{\captionfont}{\tiny}
\newcommand{\QtwoQone}{$Q_2\times Q_1$}
\newcommand{\python}{\color{darkgray} \sffamily }
\newcommand{\K}{{\mathbb{K}}}
\newcommand{\G}{{\mathbb{G}}}

%%%%%%%%%%%%%%%%%%%%%%%%%%%%%%%%%%%%%%%%%%%%%%%%%%%%%%%%%%%%%%%%%%%%%%%%%%%%%%%
%%%%%%%%%%%%%%%%%%%%%%%%%%%%%%%%%%%%%%%%%%%%%%%%%%%%%%%%%%%%%%%%%%%%%%%%%%%%%%%
%%%%%%%%%%%%%%%%%%%%%%%%%%%%%%%%%%%%%%%%%%%%%%%%%%%%%%%%%%%%%%%%%%%%%%%%%%%%%%%
%%%%%%%%%%%%%%%%%%%%%%%%%%%%%%%%%%%%%%%%%%%%%%%%%%%%%%%%%%%%%%%%%%%%%%%%%%%%%%%
%%%%%%%%%%%%%%%%%%%%%%%%%%%%%%%%%%%%%%%%%%%%%%%%%%%%%%%%%%%%%%%%%%%%%%%%%%%%%%%
\begin{document}
\title{FLAPS \\ the FLexible Axisymmetric Planet Solver}
\author{C. Thieulot}
\maketitle

%%%%%%%%%%%%%%%%%%%%%%%%%%%%%%%%%%%%%%%%%%%%%%%%%%%%%%%%%%%%%%%%%%%%%%%%%%%%%%%
\section{Introduction}



%%%%%%%%%%%%%%%%%%%%%%%%%%%%%%%%%%%%%%%%%%%%%%%%%%%%%%%%%%%%%%%%%%%%%%%%%%%%%%%
\newpage
\section{The physics}

%----------------------------------------------------------
\subsection{Axisymmetric formulation}


We here rely on axisymmetric cylindrical coordinates, see fieldstone. 
As shown on the following figure we assume that the deformation/flow is independent of the angle 
$\theta$ so that the remaining space coordinates are $r$ and $z$.
\begin{center}
\input{tikz/tikz_axi}
\end{center}

Given the symmetry of the problem any term containing $\partial_\theta$ or $\upnu_\theta$ is zero.
The strain rate tensor\footnote{\url{https://engineering.stackexchange.com/questions/41178/infinitesimal-strain-tensor-in-cylindrical-coordinates}} then simplifies to:

\begin{eqnarray}
\dot\varepsilon_{rr} 
&=& \frac{\partial \upnu_r}{\partial r} \nn\\
\dot\varepsilon_{\theta\theta}  &=& \frac{\upnu_r}{r} \nn\\
\dot\varepsilon_{\theta r} = \dot\varepsilon_{r\theta}  &=& 0 \nn\\
\dot\varepsilon_{zz} &=& \frac{\partial \upnu_z}{\partial z} \nn\\
\dot{\varepsilon}_{rz} = \dot{\varepsilon}_{zr} 
&=& \frac{1}{2}\left( \frac{\partial \upnu_r}{\partial z} + \frac{\partial \upnu_z}{\partial r} \right) \nn\\
\dot{\varepsilon}_{\theta z} = \dot{\varepsilon}_{z \theta} &=& 0 \nn
\end{eqnarray}
or,
\[
\dot{\bm\varepsilon}(\vec\upnu)
=
\left(
\begin{array}{ccc}
\dot\varepsilon_{rr} & 0 & \dot{\varepsilon}_{rz} \\
0 & \dot{\varepsilon}_{\theta\theta}  & 0 \\
\dot{\varepsilon}_{zr} & 0 & \dot\varepsilon_{zz}
\end{array}
\right)
\]
Note that the term $\dot\varepsilon_{\theta\theta} $ is not zero!
The deviatoric stress tensor ${\bm \tau}=2\eta \dot{\bm \varepsilon}$ can be computed
as well as the full stress tensor ${\bm \sigma}=-p {\bm 1} + {\bm \tau}$. 




%----------------------------------------------------------
\subsection{Dynamic topography}


Looking at the ASPECT manual, we find:

``we evaluate the stress and evaluate the component of it in the direction in which 
gravity acts. In other words we compute 
\[
\sigma_{rr}={\hat g}^T (2 \eta \varepsilon(\mathbf u)
- \frac 13 (\textrm{div}\;\mathbf u)I)\hat g - p_d
\] 
where 
$\hat g = \mathbf g/\|\mathbf g\|$ is the direction of 
the gravity vector $\mathbf g$ and $p_d=p-p_a$ is the dynamic 
pressure computed by subtracting the adiabatic pressure $p_a$ 
from the total pressure $p$ computed as part of the Stokes 
solve. From this, the dynamic 
topography is computed using the formula 
\[
h=\frac{\sigma_{rr}}{(\mathbf g \cdot \mathbf n)  \rho}
\] 
where $\rho$ is the density. For the bottom surface we chose the convection 
that positive values are up (out) and negative values are in (down), analogous to 
the deformation of the upper surface. 
The file format then consists of lines with Euclidean coordinates 
followed by the corresponding topography value.''

Here the viscosity is constant in the domain, the flow 
is incompressible and $\vec{g}= -g_0 \vec{e}_r$ so we can compute the 
dynamic topography as follows:
\[
h = - \frac{-p + 2 \eta_0 \dot{\varepsilon}_{rr} }{\rho_0 g_0}
\]
This is of course valid if what is above the surface is a fluid with zero density.
This translates into:

\begin{lstlisting}
dyn_topo=np.zeros(NV,dtype=np.float64)
for i in range(0,NV):
    if surfaceV[i] and xV[i]>=0:
       dyn_topo[i]= -(2*viscosity*e_rr[i]-q[i])/(rho0*g0) 
\end{lstlisting}

Note that in our case $\rho_0=1$ and $g_0=1$ so that in fact the dynamic topography 
is simply $-\sigma_{rr}$.








%%%%%%%%%%%%%%%%%%%%%%%%%%%%%%%%%%%%%%%%%%%%%%%%%%%%%%%%%%%%%%%%%%%%%%%%%%%%%%%
\newpage
\section{Numerical methods}

%----------------------------------------------------------
\subsection{Finite elements}

%----------------------------------------------------------
\subsection{Mapping}


After discretising the domain in {\python nel} elements, and having decided the FE
pair we want to use to solve the Stokes equations (in this case \QtwoQone), we end up 
having to compute elemental integrals such as 
\[
\K_e = \int_{\Omega_e} {\bm B}^T\cdot {\bm C}_\eta \cdot {\bm B} \; d\Omega
\]
where $\Omega_e$ denotes an element.
The way we carry out this integration is by means of the Gauss-Legendre quadrature, which 
forces us to carry out a change of variables from the original element $\Omega_e$ 
to the reference element $(r,s) \in [-1,1]\times [-1,1]$. For this we establish a mapping between both 
as explained in Section 7.13 of fieldstone.
Basis functions $Q_{1,2,3,4}$ are defined in Section 3.4.

\begin{center}
\includegraphics[width=4.2cm]{images/mappingQ1}
\includegraphics[width=4.2cm]{images/mappingQ2}
\includegraphics[width=4.2cm]{images/mappingQ3}
\includegraphics[width=4.2cm]{images/mappingQ4}\\
{\captionfont Layout of the mapping nodes in element \#0 of the mesh. 
From left to right: $Q_1$, $Q_2$, $Q_3$ and $Q_4$. Rows of nodes are placed 
on concentric circles and columns of nodes are equidistant in $\theta$ space.} 
\end{center}

For each element we store the coordinates of these mapping points into two 
arrays:
\begin{lstlisting}
xmapping=np.zeros((X,nel),dtype=np.float64)
ymapping=np.zeros((X,nel),dtype=np.float64)
\end{lstlisting}
where {\python X} stands for the number of nodes for each mapping.

The reduced coordinates for the quadrature points are given by 
the Gauss-Legendre quadrature approach. The real coordinates of these points
is a function of the mapping used so that 
\begin{lstlisting}
for iel in range(0,nel):
    for kq in range(0,nqel):
        rq=qcoords_r[kq]
        sq=qcoords_s[kq]
        NNNV=NNN(rq,sq,mapping)
        xq=np.dot(NNNV[:],xmapping[:,iel])
        yq=np.dot(NNNV[:],ymapping[:,iel])
\end{lstlisting}
Likewise the Jacobian matrix is by definition a function of the chosen mapping 
so that 
\begin{lstlisting}
for iel in range(0,nel):
    for kq in range(0,nqel):
        rq=qcoords_r[kq]
        sq=qcoords_s[kq]
        dNNNVdr=dNNNdr(rq,sq,mapping)
        dNNNVds=dNNNds(rq,sq,mapping)
        jcb[0,0]=np.dot(dNNNVdr[:],xmapping[:,iel])
        jcb[0,1]=np.dot(dNNNVdr[:],ymapping[:,iel])
        jcb[1,0]=np.dot(dNNNVds[:],xmapping[:,iel])
        jcb[1,1]=np.dot(dNNNVds[:],ymapping[:,iel])
        jcob=np.linalg.det(jcb)
        jcbi=np.linalg.inv(jcb)
\end{lstlisting}


\begin{center}
\includegraphics[width=4.2cm]{images/nq4}
\includegraphics[width=4.2cm]{images/nq9}
\includegraphics[width=4.2cm]{images/nq16}
\includegraphics[width=4.2cm]{images/nq25}\\
{\captionfont Layout of the quadrature points in element \#0 of the mesh. 
From left to right: {\python nqperdim=2,3,4,5}.} 
\end{center}

Note that the {\python axisymmetric} flag controls whether the Stokes equations 
are solved in plane strain or under the assumption that there is axisymmetry. 
In the latter case the mesh is a demi-annulus in the $x>0$ half plane.



%----------------------------------------------------------
\subsection{Quadrature}



%----------------------------------------------------------
\subsection{Computing normals}

%----------------------------------------------------------
\subsection{Free slip}


Finally free slip boundary conditions have been implemented, but only at the 
surface, and only with the method of Lagrange Multipliers (stone~151
taught us that it works as well as the other method).  
{\color{red} change}

\begin{eqnarray}
\K_e \cdot \vec{\cal V} + \G_e \cdot \vec{\cal P} &=& \vec{f}  \\
\G_e \cdot \vec{\cal V} &=& \vec{0}
\end{eqnarray}
We multiply the first line by the rotation matrix ${\bm R}$:
\begin{eqnarray}
{\bm R} \cdot \K_e \cdot \vec{\cal V} +{\bm R} \cdot \G_e \cdot \vec{\cal P} &=&{\bm R} \cdot \vec{f}  \\
\G_e \cdot \vec{\cal V} &=& \vec{0}
\end{eqnarray}
and then introduce the identity matrix ${\bm I}={\bm R}^T\cdot {\bm R}$ before the velocity vector:
\begin{eqnarray}
{\bm R} \cdot \K_e \cdot {\bm R}^T\cdot {\bm R} \cdot  \vec{\cal V} +{\bm R} \cdot \G_e \cdot \vec{\cal P} &=&{\bm R} \cdot \vec{f}  \\
\G_e \cdot  {\bm R}^T\cdot {\bm R} \cdot  \vec{\cal V} &=& \vec{0}
\end{eqnarray}
The second line can also be written
\begin{eqnarray}
({\bm R} \cdot \K_e \cdot {\bm R}^T) \cdot ({\bm R} \cdot  \vec{\cal V}) + ({\bm R} \cdot \G_e) \cdot \vec{\cal P} &=&{\bm R} \cdot \vec{f}  \\
( {\bm R} \cdot \G_e)^T \cdot  ({\bm R} \cdot  \vec{\cal V}) &=& \vec{0}
\end{eqnarray}
which translates at the elemental level into
\begin{lstlisting}
K_el=RotMat.dot(K_el.dot(RotMat.T))
f_el=RotMat.dot(f_el)
G_el=RotMat.dot(G_el)
\end{lstlisting}
Note that the matrix $\K_e$ is $(m*ndofV)\times (m*ndofV)$ in size, and so is the matrix ${\bm R}$.

After boundary conditions are imposed, the system is rotated back:

\begin{lstlisting}
K_el=RotMat.T.dot(K_el.dot(RotMat))
f_el=RotMat.T.dot(f_el)
G_el=RotMat.T.dot(G_el)
\end{lstlisting}


%----------------------------------------------------------
\subsection{Pressure normalisation}

%-------------------------
\subsubsection*{Plane strain}

In polar coordinates the surface element is 
\[
dS = R d\theta
\]
so that 
$\int_0^{2\pi} dS = \int_0^{2\pi} R d\theta = 2 \pi R$ which is the perimeter of the circle.

We wish to normalise the pressure so that it is on average zero on the surface:
\[
p_{normalised} = p - <p>
\]
where 
\[
<p> 
= \frac{\int p(r, \theta) dS }{\int dS}
= \frac{\int p(\theta) R d\theta }{\int R d \theta d\theta}
= \frac{1}{2 \pi R} R \int p(\theta) d\theta 
= \frac{1}{2 \pi }  \int_0^{2\pi} p(\theta) d\theta 
\]
This integral is broken up in a summation over element edges which are at the surface.
\[
<p>= \frac{1}{2\pi} \sum_{e}^{nelt} \int_{\theta_{2}^e}^{\theta_{3}^e} p (\theta) d\theta
\]
where $\theta_2^e$ and $\theta_3^e$ are the $\theta$ values of nodes 2 and 3 of element $e$ that lie on the surface.
These edge integrals are simplifed by assuming a 1-point quadrature:
\[
<p>= \frac{1}{2\pi} \sum_{e}^{nelt}  p (\frac{\theta_2^e+\theta_3^e}{2}) (\theta_3^e-\theta_2^e)
\]



%------------------------------
\subsubsection*{Axisymmetric case}

\begin{center}
\includegraphics[width=3cm]{images/sphcoord}
\end{center}

In spherical coordinates, the surface element is 
\[
dS= R^2 \sin \theta d\theta d\phi
\]
We wish to normalise the pressure so that it is on average zero on the surface:
\[
p_{normalised} = p - <p>
\]
where 
\[
<p> 
= \frac{\iint p(\theta,\phi) dS }{\iint dS}
= \frac{\iint p(\theta,\phi) R^2 \sin \theta d\theta d\phi}{\iint R^2 \sin \theta d\theta d\phi}
\]
and since $p$ is independent of $\phi$ then 
\[
<p> 
= \frac{R_2^2 \cdot 2\pi \cdot  \int_0^\pi p(\theta)  \sin \theta d\theta}
{R_2^2 \cdot 2\pi \cdot \int  \sin \theta d\theta }
= \frac{ 2\pi R_2^2 \int_0^\pi p(\theta)  \sin \theta d\theta} {4\pi R_2^2}
\]
The integral over $\theta$ can be simplified by using the average pressure 
along the edge and the angle of the edge middle point:
\begin{lstlisting}
poffset=0
for iel in range(0,nel):
    if surface_element[iel]:
       dtheta=theta_sph[iconV[2,iel]]-theta_sph[iconV[3,iel]]
       pmean=0.5*(p[iconP[2,iel]]+p[iconP[3,iel]])
       poffset+=np.sin((theta_sph[iconV[2,iel]]+theta_sph[iconV[3,iel]])/2)*dtheta\
                *2*np.pi*R2**2 * pmean
poffset/=4*np.pi*R2**2
\end{lstlisting}






%%%%%%%%%%%%%%%%%%%%%%%%%%%%%%%%%%%%%%%%%%%%%%%%%%%%%%%%%%%%%%%%%%%%%%%%%%%%%%%
\newpage
\section{The data}

%----------------------------------------------------------
\subsection{Earth}

We assume that viscosity is purely a function of depth\footnote{This is borrowed
from stone 71}.. 

Five radial viscosity profiles are available:
\begin{itemize}
\item The first viscosity profile is a constant viscosity for all depths of $10^{22}$ Pa s.
This value is an estimated value of what is normally found in the literature. 

\item The second viscosity profile comes from Yoshida et al (2001) \cite{yohk01}. It uses three different regions: lithosphere (0 km to 150 km), upper mantle (150 km to 670 km) and lower mantle (670 km to 2900 km).

\item The third viscosity profile comes from Steinberger \& Holmes (2008) \cite{stho08}
which is comparable to \cite{stca06}, but of the latter no available data was available.
Data is read from the file \texttt{DATA/EARTH/eta\_stho08.ascii}.

\item The fourth and fifth profile come from Ciskova et al (2012) \cite{civs12}.
Data is read from the file \texttt{DATA/EARTH/eta\_civs.ascii}.
The paper showcases two main families of radial viscosity profiles in literature. Family A, which has a sharp
increase below the 660 km transition zone and remains constant for most of the lower mantle
and family B which is much smoother over the transition zone and increases with depth in the lower mantle.

\end{itemize}

Three radial density profiles are available:

\begin{itemize}
\item PREM \cite{dzan81}
\item ak135f \cite{keeb95} \url{http://rses.anu.edu.au/seismology/ak135/ak135f.html}
Data is read from the file \texttt{DATA/EARTH/rho\_ak135f.ascii}.
\item stw105 \cite{kued08} \url{http://ds.iris.edu/ds/products/emc-stw105/}
Data is read from the file \texttt{DATA/EARTH/rho\_stw105.ascii}.
\end{itemize}



%----------------------------------------------------------
\subsection{Mars}


%%%%%%%%%%%%%%%%%%%%%%%%%%%%%%%%%%%%%%%%%%%%%%%%%%%%%%%%%%%%%%%%%%%%%%%%%%%%%%%
\newpage
\section{Benchmarking}



The goal here is to explore the influence of the mapping polynomial order and/or
the number of quadrature points on the accuracy of the solution of various benchmarks and test cases.

Concretely, in this section we will explore the effect of:
\begin{itemize}
\item resolution via the number of elements in the radial direction: {\python nelr=2-32} (we automatically set {\python nelt=12*nelr})
\item the number of quadrature points per dimension: {\python nqperdim=2,3,4,5}
\item the polynomial order of the mapping: {\python mapping='Q1','Q2','Q3','Q4'}
\end{itemize}
and we will monitor the computed area/volume, the root mean square velocity and the velocity and pressure errors.

\subsection{Computing volume/mass}


\subsection{Annulus convection manufactured solution}


%%%%%%%%%%%%%%%%%%%%%%%%%%%%%%%%%%%%%%%%%%%%%%%%%%%%%%%%%%%%%%%%%%%%%%%%%%%%%%%
\section{The '4D dynamic earth' inter-code benchmark}

\cite{krhb12}


%%%%%%%%%%%%%%%%%%%%%%%%%%%%%%%%%%%%%%%%%%%%%%%%%%%%%%%%%%%%%%%%%%%%%%%%%%%%%%%
%%%%%%%%%%%%%%%%%%%%%%%%%%%%%%%%%%%%%%%%%%%%%%%%%%%%%%%%%%%%%%%%%%%%%%%%%%%%%%%
\newpage
\appendix
\section{Misc}

%----------------------------------------------------------
\subsection{Notes to self}

What I have tried to cure the pb of the weird anomalies at the poles.

\begin{itemize}
\item turning elements into real trapezes. Made things worse
\item different mappings. not much difference
\item when using blob, reduced densities. no difference
\item nb of quad points, no real difference
\item nb of elements in tangential direction, some difference but no cure 
\item when using blob, drho/rho, no diff 
\item type of b.c. at point corner below poles, no real diff 
\item scaling of G matrix
\item different rotations/bc for free slip, no difference
\item using cmat matrix for dev strain rate, helped a little bit, no cure 
\end{itemize}



%----------------------------------------------------------
\subsection{To do list}
\begin{itemize}
\item visc profiles
\item rho profiles
\item time stepping
\item gravity calculations. import from f96, re-benchmark
\item CBF? 
\item compute self gravity for reduced density case 
\item export exx1 and exx3 to outside function, clean their code too? 
\item remove call to math 
\item bottom free slip 
\item change y for z in stone
\item use PREM gravity value
\item aspect with GMG ?
\item compute moment of inertia
\item by default code now uses elemental rho and eta. it changes things wrt exp0 benchmark results!
\end{itemize}




\bibliography{bibliography} 

\end{document}
%%%%%%%%%%%%%%%%%%%%%%%%%%%%%%%%%%%%%%%%%%%%%%%%%%%%%%%%%%%%%%%%%%%%%%%%%%%%%%%
