There are three different methods implemented in the code.
After various tests it was found that method 2 was the 
cheapest and most accurate. The other two remain in the code
but are switched off by default.

\begin{enumerate}
\item This is implemented in {\python compute\_strain\_rate1}. The call to this function 
is triggered by the value of {\python compute\_sr1}. The strain rate components are 
first computed in the middle of each element {\python exxc,eyyc,exyc} and these quantities 
are then averaged onto nodes. This yields the nodal quantities {\python exx1,eyy1,exy1}. 
\item This is implemented in {\python compute\_strain\_rate2}. Inside each element 
the strain rate components are computed/added at the V nodes. In the end average 
nodal values are arrived at and yield {\python exx2,eyy2,exy2}.
  
\item This is implemented in {\python compute\_strain\_rate3}.
The call to this function is triggered by the value of {\python compute\_sr3}. 
This method computes the velocity gradient components at the nodes by solving 
a linear system\footnote{I need to find the ref for this method in FS}.
It yields yield {\python exx3,eyy3,exy3}.

\end{enumerate}

After each strain rate tensor is computed its effective value is computed, e.g.:
\begin{lstlisting}
sr2=np.zeros(NV,dtype=np.float64)
sr2[:]=np.sqrt(0.5*(exx2[:]**2+eyy2[:]**2)+exy2[:]**2)
\end{lstlisting}
which is the square root of the second invariant.


