
Five radial viscosity profiles are available (see {\tt viscosity\_Earth.py}):
\begin{itemize}
\item
{\python eta\_model=0}:
 The first viscosity profile is a constant viscosity for all depths of $\eta_m$. 

\item
{\python eta\_model=1,2}:
 The fourth and fifth profile come from Ciskova et al (2012) \cite{civs12}.
Data is read from the file \texttt{DATA/EARTH/eta\_civs.ascii}.
The paper showcases two main families of radial viscosity profiles in literature. Family A, which has a sharp
increase below the 660 km transition zone and remains constant for most of the lower mantle
and family B which is much smoother over the transition zone and increases with depth in the lower mantle.

\item 
{\python eta\_model=3}:
The third viscosity profile comes from Steinberger \& Holmes (2008) \cite{stho08}
which is comparable to \cite{stca06}, but of the latter no available data was available.
Data is read from the file \texttt{DATA/EARTH/eta\_stho08.ascii}.

\item
{\python eta\_model=4}:
 The second viscosity profile comes from Yoshida et al (2001) \cite{yohk01}. It uses three different regions: lithosphere (0 km to 150 km), upper mantle (150 km to 670 km) and lower mantle (670 km to 2900 km).

\end{itemize}

Three radial density profiles are available (see {\tt density\_Earth.py}):

\begin{itemize}
\item
{\python rho\_model=0}: density is constant and set to $\rho_m$.

{\python rho\_model=1}:
 PREM \cite{dzan81}

\item
{\python rho\_model=2}
 stw105 \cite{kued08} \url{http://ds.iris.edu/ds/products/emc-stw105/}
Data is read from the file \texttt{DATA/EARTH/rho\_stw105.ascii}.

\item
{\python rho\_model=3}
 ak135f \cite{keeb95} \url{http://rses.anu.edu.au/seismology/ak135/ak135f.html}
Data is read from the file \texttt{DATA/EARTH/rho\_ak135f.ascii}.


\end{itemize}


The buoyant object is a sphere of radius $R_s$, placed at $z=z_s$ above the 
center of the planet. It is parameterised by $\rho_{blob}^\star =\rho_{blob}/$ 

