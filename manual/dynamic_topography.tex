

Looking at the ASPECT manual, we find:

``we evaluate the stress and evaluate the component of it in the direction in which 
gravity acts. In other words we compute 
\[
\sigma_{rr}={\hat g}^T (2 \eta \varepsilon(\mathbf u)
- \frac 13 (\textrm{div}\;\mathbf u)I)\hat g - p_d
\] 
where 
$\hat g = \mathbf g/\|\mathbf g\|$ is the direction of 
the gravity vector $\mathbf g$ and $p_d=p-p_a$ is the dynamic 
pressure computed by subtracting the adiabatic pressure $p_a$ 
from the total pressure $p$ computed as part of the Stokes 
solve. From this, the dynamic 
topography is computed using the formula 
\[
h=\frac{\sigma_{rr}}{(\mathbf g \cdot \mathbf n)  \rho}
\] 
where $\rho$ is the density. For the bottom surface we chose the convection 
that positive values are up (out) and negative values are in (down), analogous to 
the deformation of the upper surface. 
The file format then consists of lines with Euclidean coordinates 
followed by the corresponding topography value.''

Here the viscosity is constant in the domain, the flow 
is incompressible and $\vec{g}= -g_0 \vec{e}_r$ so we can compute the 
dynamic topography as follows:
\[
h = - \frac{-p + 2 \eta_0 \dot{\varepsilon}_{rr} }{\rho_0 g_0}
\]
This is of course valid if what is above the surface is a fluid with zero density.
This translates into:

\begin{lstlisting}
dyn_topo=np.zeros(NV,dtype=np.float64)
for i in range(0,NV):
    if surfaceV[i] and xV[i]>=0:
       dyn_topo[i]= -(2*viscosity*e_rr[i]-q[i])/(rho0*g0) 
\end{lstlisting}

Note that in our case $\rho_0=1$ and $g_0=1$ so that in fact the dynamic topography 
is simply $-\sigma_{rr}$.





